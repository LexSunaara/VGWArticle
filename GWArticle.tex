\documentclass[english]{article}
\usepackage[T1]{fontenc}
\usepackage[latin9]{inputenc}
\usepackage{amsmath}
\usepackage{amssymb}
\usepackage[margin=2cm]{geometry}

\usepackage{babel}
\begin{document}

\title{Generation of viscous internal gravity waves by a moving boundary}
\author{A. Renaud and A. Venaille}
\maketitle

\section{Introduction}

Internal gravity waves are a field of increasing importance in fluid mechanics. The underlying reason is the major role they play in many crucial phenomena appearing in the ocean and the atmosphere where the stratification cannot be ignored. They have been extensively studied theoretically, numerically and experimentally. The experimental studies of internal gravity waves have made great progress recently. It started from the pioneer work of Long and Baines in 1955 and 1977 \cite{Long1955,Baines1977} where the waves are generated by draging a smooth bump at the bottom of a tank or Plumb and McEwan in 1978 \cite{Plumb1978} where a standing wave is excited by pistons making a rubber membrane oscillating at the bottom of a annular tank. In 2006, Gostiaux et al \cite{Gostiaux2006} proposed a new generator of internal gravity waves allowing for excitation of propagating monochromatic waves. This last technique led to a new class of internal gravity waves experiments (see \cite{Peacock2009,Gostiaux2007}). The generation by drag is still used in recent studies (see e.g. \cite{Dossmann2016}).

The inviscid problem has been extensively studied (see \cite{Muraschko2015}). If the inviscid theory is relevant in many oceanic and atmospheric problems, the large difference in spatial and time scaling between the geophysical internal gravity waves and their laboratory analogues makes it dodgy to ignore the dissipative processes. The theory of the quasi\--biennial oscillation occuring in the equatorial stratosphere needs a dissipative process to attenuate the waves beams coming from tropical troposphere (see \cite{Lindzen1968,Plumb1977}). In the atmospheric context, this dissipation comes mainly from the Newtonian cooling parametrised by a drag force in the buoyancy euqation (see e.g. \cite{Grisouard2012}). However, in the laboratory analogue performed by \cite{Plumb1978} the dominant dissipation process comes from the viscosity. The main difference between these two features is that the latter one involves viscous boundary layers to match the propagating waves with the boundary condition at level of the waves generator. At quasi\--linear level those boundary layers can involve an important streaming close to the generator. 

In high\--resolution numerical simulations of stratified fluids motion over topography, the sub\--grid scale dynamics is often parametrized by an effective viscosity several order of magnitude higher than the actual viscosities involved in the simulated phenomena. Therefore, viscous effects might have to be taken in account, especially if the resolution is high enough compared to the viscous boundary layer length scale, even if a free\--slip boundary condition is chosen for the velocity field.

Plan du papier :
\begin{itemize}
	\item Section 1 : Modele, equation primitive, approximation de boussinesq
	\item Section 2 : Zonally symetric waves and mean\--flow interaction
	\item Section 3 : WKB asymptotics and wave properties
	\item Section 4 : Boundary condition and reynold stress
	\item Section 5 : Applications : Lee waves, QBO Plumb
	\item Section 6 : Simulations
	\item Section 7 : Discussion/Conclusion
\end{itemize}





\begin{table}
	\begin{tabular}{|c|c|c|c|c|c|c|}
	\hline 
	\textbf{Studies} 	& $\mathrm{Re}$	& $\mathrm{Fr}$	& $\epsilon$	& $\mathrm{Pr}$	& Generation			& Type\tabularnewline
	\hline 
	\hline 
	\cite{Baines1985} 	& $100-1000$ 	& $0.1$ 		& $0.1$ 		& $1000$ 		& ridge drag 			& Experimental\tabularnewline
	\hline 
	\cite{Dossmann2016} & $400-87000$ 	& $0.1-34$ 		& $0.1-1.$ 		& $1000$ 		& ridge drag 			& Experimental\tabularnewline
	\hline 	
	\cite{Peacock2009} 	& $73$ 			& $0.82$ 		& $0.4$ 		& $1000$ 		& oscillating cams 		& Experimental\tabularnewline
	\hline 
	\cite{Plumb1978} 	& $1500-3000$ 	& $0.02-0.04$ 	& $0.05$ 		& $1000$ 		& oscillating membranes & Experimental\tabularnewline
	\hline 
	\cite{Semin2016} 	& $650$ 		& $0.1 $		& $0.06$ 		& $1000$ 		& oscillating membranes & Experimental\tabularnewline
	\hline 
\end{tabular}

\caption{Referencing of some characteristic dimensionless number in internal gravity wave experiments.\label{tab:Referencing-of-some} }
\end{table}


\section{The Boussinesq model}

We consider a stratified fluid within a  2\--dimensional domain periodic in the horizontal $x$\--direction (this direction will be referred to as ``zonal'' in the remaining of the paper) and semi-infinite in the vertical $z$\--direction. We consider the incompressible Boussinesq model. Two types of dissipative mechanism will be considered: viscosity and Newtonian cooling. The equation of state is linear. The bottom boundary is an undulating line located on average at $z=0$. We are interested in the dynamics of the buoyancy anomaly defined by $b=g\frac{\left(\rho_{c}-\rho\right)}{\rho_{c}}-\int^{z}N^{2}$ where $\rho_{c}$ is a reference value for the density field $\rho$ and $N$ is the Br�nt-V�is�l� frequency (height\--dependent in general and always positive such that the rest state, $b=0$, is stable). The parameters of the problem are the characteristic amplitude, wave number and angular momentum of the bottom boundary undulation $\left(h_{b},k,\omega\right)$, the kinematic viscosity $\nu$ and the bottom value of the Br�nt\--V�is�l� frequency $N_{0}=N\left(z=0\right)$. We denote $c=\omega/k$ the associated phase\--speed. We can define three independent dimensionless parameters:
\begin{equation}
	\begin{cases}
		\text{Reynold number :} 			& \mathrm{Re}=\frac{c}{\nu k}\\
		\text{Froude number : } 			& \mathrm{Fr}=\frac{\omega}{N_{0}}\\
		\text{Bottom boundary mean slope :}	& \epsilon=h_{b}k
	\end{cases}.
\end{equation}

\subsection{Equations}

The flow obeys the primitive equations in the Boussinesq approximation. In the adimensionalized form, they writes:
\begin{equation}
	\begin{cases}
		\partial_{t}u+u\partial_{x}u+w\partial_{z}u 	& =-\partial_{x}P+\frac{1}{\mathrm{Re}}\Delta u\\
		\partial_{t}w+u\partial_{x}w+w\partial_{z}w 	& =-\partial_{z}P+\frac{1}{\mathrm{Fr}^{2}}b+\frac{1}{\mathrm{Re}}\Delta w\\
		\partial_{t}b+u\partial_{x}b+w\partial_{z}b+n\left(z\right)w 	& =-\alpha b\\
		\partial_{x}u+\partial_{z}w 					& =0
	\end{cases}\label{eq:GlobalEQ_NonLin}
\end{equation}
where  $\Delta=\partial_{xx}+\partial_{zz}$ is the Laplacian differential operator , $\left(u,w\right)$ are the components of the 2\--dimensional velocity vector field, $P$ is the pressure anomaly field, $b$ is the buoyancy anomaly field, $n$ is the rescaled Br�nt-V�is�l� frequency field ($n\left(0\right)=1$ and $n\left(z>0\right)=1$ if the stratification is linear) and $\alpha$ is a parameter parameterizing the relaxation of the buoyancy anomaly toward the rest state.  The first equation will be referred to as the zonal momentum equation, the second one as the vertical momentum equation, the third one as the buoyancy equation and the last one as the incompressibility condition. 

\subsubsection*{Boundary conditions}

The domain is $2\pi$\--periodic in the zonal direction. The equation of the bottom boundary is given by $z\left(x,t\right)=\epsilon h_{b}\left(x,t\right)$ with $\int_{0}^{2\pi}h_{b}\left(x,t\right)\mathrm{d}x=0$. All the following equation for the boundary conditions have to be evaluated in $z= \epsilon h_{b}\left(x,t\right)$.

If $\mathrm{Re}=\infty$, we consider an impermeability condition at the bottom boundary. It amounts to fix the normal velocity at the boundary to be equal to its normal velocity:
\begin{equation}
	w-w_{h_{b}}-\epsilon\left(u-u_{h_{b}}\right)\partial_{x}h_{b} 	= 0 
\end{equation}
where $\left(u_{h_{b}},w_{h_{b}}\right)$ are the component of the velocity of the bottom boundary.

If $\mathrm{Re}<\infty$, we will either consider a no\--slip or a free\--slip boundary condition :
\begin{itemize}
	\item No\--slip 
	\begin{equation}
			\left(u,w\right) = \left(u_{h_{b}},w_{h_{b}}\right)
	\end{equation}
	\item Free\textendash slip 
	\begin{equation}
		\begin{cases}
			w-w_{h_{b}}-\epsilon\left(u-u_{h_{b}}\right)\partial_{x}h_{b}																													& 	= 0 	\\
			\left(\partial_{z}u+\partial_{x}w\right)\left(1-\left(\epsilon\partial_{x}h_{b}\right)^{2}\right)-2\epsilon\partial_{x}h_{b}\left(\partial_{x}u-\partial_{z}w\right)			& 	= 0
		\end{cases}
	\end{equation}
	the first equation is the impermeability condition, the second one is the condition of vanishing tangential constraint.
\end{itemize}

\section{Zonally symmetric wave\--mean flow interaction}

Let us introduce the zonal average operator:
\begin{equation}
	\overline{\phi}\left(z,t\right)=\frac{1}{2\pi}\int_{0}^{2\pi}\mathrm{d}x\;\phi\left(x,z,t\right),
\end{equation}
where $\phi$ is any field. We define the waves-mean flow decomposition by 
\begin{equation}
	\left(u,w,b,P\right)=\left(\overline{u},\overline{w},\overline{b},\overline{P}\right)+\left(u',w',b',P'\right),
\end{equation}
the fields $\left(\overline{u},\overline{w},\overline{b},\overline{P}\right)$ will be referred to as the mean\textendash flow and the fields $\left(u',w',b',P'\right)$ as the waves. 

We can notice that if there is no Newtonian cooling (i.e. $\alpha=0$ in Eq. (\ref{eq:GlobalEQ_NonLin})), any integral over the domain of the total buoyancy $b+\int^{z}n^{2}$ is conserved by the dynamics. It implies that the total buoyancy field has to be a rearrangement of its initial condition. This means that the mean buoyancy field is of order two in terms of the wave amplitude. In the following we will consider small amplitude waves. We, hereby, ignore the equation of evolution of the mean buoyancy field $\overline{b}$ and we make the approximation $\overline{b}=0$ in the following. The averaging of the continuity equation in (\ref{eq:GlobalEQ_NonLin}) leads to $\overline{w}=0$. Consequently, the averaging of the vertical momentum equation simply leads to a diagnostic equation for the mean pressure field $\overline{P}$. The equation of interest is the averaged zonal momentum equation : 
\begin{equation}
	\partial_{t}\overline{u}-\frac{1}{\mathrm{Re}}\partial_{zz}\overline{u}=-\partial_{z}\overline{u'w'}.
\end{equation}
The important term is the Reynold stress $\partial_{z}\overline{u'w'}$. The wave equation need to be solved to compute it. Subtracting the mean\--flow equations to (\ref{eq:GlobalEQ_NonLin}), we obtain the fully non\--linear wave equation:
\begin{equation}
	\begin{cases}
		\partial_{t}u'+\overline{u}\partial_{x}u'+w'\partial_{z}\overline{u}+\mathcal{B}\left(\mathbf{u}',u'\right)	& =-\partial_{x}P'+\frac{1}{\mathrm{Re}}\Delta u'\\
		\partial_{t}w'+\overline{u}\partial_{x}w'+\mathcal{B}\left(\mathbf{u}',w'\right) 							& =-\partial_{z}P'+\frac{1}{\mathrm{Fr}^{2}}b'+\frac{1}{\mathrm{Re}}\Delta w'\\
		\partial_{x}u'+\partial_{z}w' 																				& =0\\
		\partial_{t}b'+\overline{u}\partial_{x}b''+w'+\mathcal{B}\left(\mathbf{u}',b'\right)						& =-\beta b'
	\end{cases}
\end{equation}
where $\mathcal{B}\left(\mathbf{u}',\phi'\right)=u'\partial_{x}\phi'+w'\partial_{z}\phi'-\overline{u'\partial_{x}\phi'+w'\partial_{z}\phi}$ is a bilinear operator with zero zonal average.

In principle, we perform the same flow decomposition at the boundaries in order to get the boundary condition for the wave and the mean part. The resulting equations are messy in general but there expressions are greatly simplified in a far-field approximation when the domain geometry is asymptotically flat in the limit $\epsilon\ll1$. 

\section{Internal gravity waves within a weakly sheared mean-flow}

We consider here a single topographic mode $k=\pm1$ and a single phase speed $c=\pm1$. We consider the limit a small topographic aspect ratio (i.e. $\epsilon\ll1$). In this limit we can consider an asymptotically square domain with a bottom boundary condition expressed at $z=0$ in terms of a series of power of $\epsilon$:
\begin{itemize}
	\item Impermeability
	\begin{equation}
		\sum_{n=2}^{\infty}\epsilon^{n}h_{b}^{n-1}\left(h_{b}\partial_{z}^{n}w-\partial_{x}h_{b}\partial_{z}^{n-1}u\right)+w-v_{z,h_{b}}+\epsilon\left(h_{b}\partial_{z}w-\left(u-u_{x,h_{B}}\right)\partial_{x}h_{b}|_{z=\epsilon h_{b}}\right)=0.
	\end{equation}
	\item Viscous boundary conditions :
\end{itemize}

\paragraph{Free-slip : No tangential constraint}
\begin{equation}
	\sum_{n=2}^{\infty}\epsilon^{n}\left(\frac{h_{b}^{n}}{2}\left(\partial_{z}^{n+1}u+\partial_{x}\partial_{z}^{n}w\right)-\frac{\partial_{x}\left(h_{b}^{n-1}\right)\partial_{x}h_{b}}{2\left(n-1\right)}\left(\partial_{z}^{n-1}u+\partial_{x}\partial_{z}^{n-2}w\right)-\frac{\partial_{x}\left(h_{b}^{n}\right)}{n}\left(\partial_{x}\partial_{z}^{n-1}u-\partial_{z}^{n+1}w\right)\right)+\epsilon\left(\frac{h_{b}}{2}\left(\partial_{zz}u+\partial_{zx}w\right)-\partial_{x}h_{b}\left(\partial_{x}u-\partial_{z}w\right)\right)+\frac{1}{2}\left(\partial_{z}u+\partial_{x}w\right)=0.
\end{equation}


\paragraph{No-slip : Cancelation of the velocity}
\begin{equation}
	\sum_{n=1}^{\infty}\epsilon^{n}h_{b}^{n}\partial_{z}^{n}\mathbf{u}=\mathbf{v}_{h_{b}}.
\end{equation}


Let us introduce a small parameter $\epsilon^{2}\ll a\ll1$. We will see later how it scales with $\epsilon$. Let us introduce a slow variable $Z=az$. We now assume that the mean flow $\overline{u}$ depend only on $Z$. This mean that for the wave field, the mean flow appears weakly sheared and frozen in time.

Let us introduce the following WKB expansion of a stationary solution of the wave-equation
\begin{equation}
	\left[
	\begin{array}{c}
		u'\\
		w'\\
		b'\\
		p'
	\end{array}
	\right] = \epsilon\sum_{j=0}^{+\infty}a^{j}\mathbb{R}\mathrm{e}\left\{ \left[
	\begin{array}{c}
		\tilde{u}{}_{j}\left(Z\right)\\
		\tilde{w}{}_{j}\left(Z\right)\\
		\tilde{b}{}_{j}\left(Z\right)\\
		\tilde{p}_{j}\left(Z\right)
	\end{array}
	\right]\exp\left(ik\left(ct-x\right)-\frac{i\Phi\left(Z\right)}{a}\right)\right\} .
\end{equation}
Injecting this expansion into the previous equation and collecting the leading order in $a$ leads :
\begin{equation}
	\epsilon\boldsymbol{\mathrm{M}}\left[
	\begin{array}{c}
		\tilde{u}{}_{0}\\
		\tilde{w}{}_{0}\\
		\tilde{b}{}_{0}\\
		\tilde{p}{}_{0}
	\end{array}
	\right]+\epsilon a\left(\boldsymbol{\mathrm{M}}\left[
	\begin{array}{c}
		\tilde{u}{}_{1}\\
		\tilde{w}{}_{1}\\
		\tilde{b}{}_{1}\\
		\tilde{p}{}_{1}
	\end{array}
	\right]+\left[
	\begin{array}{c}
		\tilde{w}{}_{0}\partial_{Z}\overline{u}+\frac{i}{\mathrm{Re}}\left(\tilde{u}_{0}\partial_{Z}m+2m\partial_{Z}\tilde{u}_{0}\right)\\
		\partial_{Z}\tilde{P}{}_{0}+\frac{i}{\mathrm{Re}}\left(\tilde{w}_{0}\partial_{Z}m+2m\partial_{Z}\tilde{w}_{0}\right)\\
		0\\
		\partial_{Z}\tilde{w}{}_{0}
	\end{array}
	\right]\right)+o\left(\epsilon a\right)=0
\end{equation}
with 
\begin{equation}
	\boldsymbol{\mathrm{M}}=\left[
	\begin{array}{cccc}
		ik\left(c-\overline{u}\right)\left(1-i\frac{k^{2}+m^{2}}{\mathrm{Re}k\left(c-\overline{u}\right)}\right) & 0 & 0 & -ik\\
		0 & ik\left(c-\overline{u}\right)\left(1-i\frac{k^{2}+m^{2}}{\mathrm{Re}k\left(c-\overline{u}\right)}\right) & -\frac{1}{\mathrm{Fr}^{2}} & -im\\
		0 & 1 & ik\left(c-\overline{u}\right)\left(1-i\frac{\beta}{k\left(c-\overline{u}\right)}\right) & 0\\
		-ik & -im & 0 & 0
	\end{array}
	\right]\qquad\text{ and }m=\partial_{Z}\phi
\end{equation}
This implies 
\begin{equation}
	\det\boldsymbol{\mathrm{M}}=0\implies\mathrm{Fr}^{2}\left(c-\overline{u}\right)^{2}\left(1-i\frac{\left(k^{2}+m^{2}\right)}{\mathrm{Re}k\left(c-\overline{u}\right)}\right)\left(1-i\frac{\beta}{k\left(c-\overline{u}\right)}\right)\left(k^{2}+m^{2}\right)=1\qquad\text{ and }\qquad\left[
	\begin{array}{c}
		\tilde{u}{}_{0}\\
		\tilde{w}{}_{0}\\
		\tilde{b}{}_{0}\\
		\tilde{p}{}_{0}
	\end{array}
	\right]=\phi_{0}\left(Z\right)\boldsymbol{\mathcal{P}}\left(Z\right)
\end{equation}
where $\tilde{\phi}_{0}$ measures the global amplitude of the wave and $\boldsymbol{\mathcal{P}}$ is the polarization of the wave given by 
\begin{equation}
	\boldsymbol{\mathcal{P}}=\left[
		\begin{array}{c}
			c-\overline{u}\\
			-{\displaystyle \frac{k}{m}}\left(c-\overline{u}\right)\\
			{\displaystyle -\frac{i}{m\left(1-i\frac{\beta}{k\left(c-\overline{u}\right)}\right)}}\\
			\left(c-\overline{u}\right)^{2}\left(1-i\frac{\left(k^{2}+m^{2}\right)}{\mathrm{Re}k\left(c-\overline{u}\right)}\right)
		\end{array}
	\right].
\end{equation}
We now project the order one terms on
\begin{equation}
	\left[
	\begin{array}{c}
		c-\overline{u}\\
		-{\displaystyle \frac{k}{m}}\left(c-\overline{u}\right)\\
		{\displaystyle \frac{i}{\mathrm{Fr}^{2}m\left(1-i\frac{\beta}{k\left(c-\overline{u}\right)}\right)}}\\
		\left(c-\overline{u}\right)^{2}\left(1-i\frac{\left(k^{2}+m^{2}\right)}{\mathrm{Re}k\left(c-\overline{u}\right)}\right)
	\end{array}
	\right].
\end{equation}
This leads toward a messy equation for the wave\--amplitude $\phi_{0}\left(Z\right)$ :
\begin{equation}
	\frac{1}{\phi_{0}\left(c-\overline{u}\right)}\partial_{Z}\left(\phi_{0}\left(c-\overline{u}\right)\right)=\frac{2\left(k^{2}-m^{2}\right)+ik\mathrm{Re}\left(c-\overline{u}\right)}{4m^{2}\left(2\left(k^{2}+m^{2}\right)+ik\mathrm{Re}\left(c-\overline{u}\right)\right)}\partial_{Z}m^{2}
\end{equation}


\subsection{Case $\mathrm{Re=\infty}$ and $\beta>0$:}

We have 
\begin{equation}
	\mathrm{Fr}^{2}\left(c-\overline{u}\right)^{2}\left(1-i\frac{\beta}{k\left(c-\overline{u}\right)}\right)\left(k^{2}+m^{2}\right)=1.
\end{equation}
This is an 2nd-order equation. There is only one propagative branch for $m$ . In the limit of $\beta\ll1$, we have :
\begin{equation}
	m=\pm\left(\underbrace{\sqrt{\frac{1}{\left(c-\overline{u}\right)^{2}\mathrm{Fr}^{2}}-k^{2}}}_{m_{0}}+i\frac{\beta}{2\left(c-\overline{u}\right)^{3}\mathrm{Fr}^{2}\sqrt{\frac{1}{\left(c-\overline{u}\right)^{2}\mathrm{Fr}}-k^{2}}}\right).
\end{equation}
In the limit $\beta\ll1$, we also have :

\begin{equation}
	\partial_{Z}\left(\frac{\phi_{0}^{2}\left(c-\overline{u}\right)^{2}}{\sqrt{\frac{1}{\left(c-\overline{u}\right)^{2}\mathrm{Fr}^{2}}-k^{2}}}\right)=0.
\end{equation}
The vertical group velocity is defined by : 
\begin{equation}
	w_{g}\equiv\frac{k}{\partial_{c}\mathbb{Re}\left[m\right]}\protect\underset{\beta\to0}{\sim}\mp\frac{km_{0}}{k^{2}+m_{0}^{2}}\left(c-\overline{u}\right)
\end{equation}


\subsection{Case $\beta=0$ and $\mathrm{Re}<\infty$ :}

We have 
\begin{equation}
	\mathrm{Fr}^{2}\left(c-\overline{u}\right)^{2}\left(1-i\frac{\left(k^{2}+m^{2}\right)}{\mathrm{Re}k\left(c-\overline{u}\right)}\right)\left(k^{2}+m^{2}\right)=1.
\end{equation}
This an 4th\-- order equation. There is two branches for $m$. A propagative one and a boundary layer one. In the limit of $\mathrm{Re}\gg1$, we have :
\begin{itemize}
	\item if $k^{2}\left(c-\overline{u}\right)^{2}\mathrm{Fr}^{2}<1$:
	\begin{align*}
		m_{\mathrm{Prop}} 	& =\pm\left(\sqrt{\frac{1}{\left(c-\overline{u}\right)^{2}\mathrm{Fr^{2}}}-k^{2}}+\frac{i}{2k\left(c-\overline{u}\right)^{3}\mathrm{Fr}^{2}\mathrm{Re}\sqrt{\frac{1}{\left(c-\overline{u}\right)^{2}\mathrm{Fr}^{2}}-k^{2}}}\right)+o\left(\mathrm{Re}^{-1}\right)\\
		m_{\mathrm{bl}} 	& =\pm\left(1-\mathrm{sign}\left(k\right)i\right)\sqrt{\frac{\left|c-\overline{u}\right|\left|k\right|\mathrm{Re}}{2}}+o\left(\sqrt{\mathrm{Re}}\right)
	\end{align*}
	\begin{align*}
		m_{\mathrm{Prop}}^{2} 	& =\frac{1}{\left(c-\overline{u}\right)^{2}\mathrm{Fr}^{2}}-k^{2}+\frac{i}{\left(c-\overline{u}\right)^{5}\mathrm{Fr}^{4}k\mathrm{Re}}+o\left(\mathrm{Re}^{-1}\right)\\
		m_{\mathrm{bl}}^{2} 	& =-i\left(c-\overline{u}\right)k\mathrm{Re}+o\left(\mathrm{Re}\right)
	\end{align*}
	Then for $m_{\mathrm{Prop}}$, we have at leading order in $\mathrm{Re}$:
	\begin{equation}
		\partial_{Z}\left(\frac{\phi_{0,\mathrm{Prop}}^{2}\left(c-\overline{u}\right)^{2}}{\sqrt{\frac{1}{\left(c-\overline{u}\right)^{2}\mathrm{Fr}^{2}}-k^{2}}}\right)=0
	\end{equation}
	and for $m_{\mathrm{bl}}$, we have at leading order in $\mathrm{Re}$:
	\begin{equation}
		\partial_{Z}\left(\phi_{0,\mathrm{bl}}m_{\mathrm{bl}}\right)=0
	\end{equation}
	\item if $k^{2}\left(c-\overline{u}\right)^{2}\mathrm{Fr}^{2}>1$:
	\begin{align*}
		m_{\mathrm{Prop}} 	& =\pm\left(i\sqrt{k^{2}-\frac{1}{\left(c-\overline{u}\right)^{2}\mathrm{Fr^{2}}}}+\frac{1}{2k\left(c-\overline{u}\right)^{3}\mathrm{Fr}^{2}\mathrm{Re}\sqrt{k^{2}-\frac{1}{\left(c-\overline{u}\right)^{2}\mathrm{Fr^{2}}}}}\right)+o\left(\mathrm{Re}^{-1}\right)\\
		m_{\mathrm{bl}} 	& =\pm\left(1-\mathrm{sign}\left(k\left(c-\overline{u}\right)\right)i\right)\sqrt{\frac{\left|c-\overline{u}\right|\left|k\right|\mathrm{Re}}{2}}+o\left(\sqrt{\mathrm{Re}}\right)
	\end{align*}
	\begin{align*}
		m_{\mathrm{Prop}}^{2}	& =\frac{1}{\left(c-\overline{u}\right)^{2}\mathrm{Fr}^{2}}-k^{2}+\frac{i}{\left(c-\overline{u}\right)^{5}\mathrm{Fr}^{4}k\mathrm{Re}}+o\left(\mathrm{Re}^{-1}\right)\\
		m_{\mathrm{bl}}^{2} 	& =-i\left(c-\overline{u}\right)k\mathrm{Re}+o\left(\mathrm{Re}\right)
	\end{align*}
	Then for $m_{\mathrm{Prop}}$, we have at leading order in $\mathrm{Re}$:
	\begin{equation}
		\partial_{Z}\left(\frac{\phi_{0,\mathrm{Prop}}^{2}\left(c-\overline{u}\right)^{2}}{\sqrt{k^{2}-\frac{1}{\left(c-\overline{u}\right)^{2}\mathrm{Fr}^{2}}}}\right)=0
	\end{equation}
	and for $m_{\mathrm{bl}}$, we have at leading order in $\mathrm{Re}$:
	\begin{equation}
		\partial_{Z}\left(\phi_{0,\mathrm{bl}}m_{\mathrm{bl}}\right)=0
	\end{equation}
\end{itemize}
\bibliographystyle{plain}
\bibliography{Bibliographie}

\end{document}
