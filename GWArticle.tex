\documentclass[english]{article}
\usepackage[T1]{fontenc}
\usepackage[latin9]{inputenc}
\usepackage{amsmath}
\usepackage{amssymb}

\usepackage{babel}
\begin{document}

\title{Generation of viscous gravity waves by a moving boundary}
\author{A. Renaud and A. Venaille}
\maketitle

\section{Introduction}
\begin{table}
	\begin{tabular}{|c|c|c|c|c|c|c|}
	\hline 
	\textbf{Studies} 	& $\mathrm{Re}$	& $\mathrm{Fr}$	& $\epsilon$	& $\mathrm{Pr}$	& Generation			& Type\tabularnewline
	\hline 
	\hline 
	\cite{Baines1985} 	& $100-1000$ 	& $0.1$ 		& $0.1$ 		& $1000$ 		& ridge drag 			& Experimental\tabularnewline
	\hline 
	\cite{Dossmann2016} & $400-87000$ 	& $0.1-34$ 		& $0.1-1.$ 		& $1000$ 		& ridge drag 			& Experimental\tabularnewline
	\hline 	
	\cite{Peacock2009} 	& $73$ 			& $0.82$ 		& $0.4$ 		& $1000$ 		& oscillating cams 		& Experimental\tabularnewline
	\hline 
	\cite{Plumb1978} 	& $1500-3000$ 	& $0.02-0.04$ 	& $0.05$ 		& $1000$ 		& oscillating membranes & Experimental\tabularnewline
	\hline 
	\cite{Semin2016} 	& $650$ 		& $0.1 $		& $0.06$ 		& $1000$ 		& oscillating membranes & Experimental\tabularnewline
	\hline 
\end{tabular}

\caption{Referencing of some characteristic dimensionless number in internal wave experiments.\label{tab:Referencing-of-some} }
\end{table}


\section{The Boussinesq model}

We consider a 2\-- dimensional linearly stratified fluid in a channel type periodic domain. The domain is semi-infinite in the vertical direction. The simplest model we need to consider is the incompressible Boussinesq model. Two type of dissipative mechanism will be considered: viscosity and Newtonian cooling. The equation of state is linear. 

\subsection{Equations}

The flow obeys the Navier-stokes equations in the Boussinesq approximation : 
\begin{equation}
	\begin{cases}
		\partial_{t}u+u\partial_{x}u+w\partial_{z}w 		& =-\partial_{x}P+\nu\Delta u\\
		\partial_{t}w+u\partial_{x}w+w\partial_{z}w 		& =-\partial_{z}P+b+\nu\Delta w\\
		\partial_{t}b+u\partial_{x}b+w\partial_{z}b+N^{2}b 	& =-\alpha b\\
		\partial_{x}u+\partial_{z}w 						& =0
	\end{cases},\label{eq:Main_Boussinesq}
\end{equation}
where $\left(x,z\right)$ are the zonal and vertical coordinates, $\Delta=\partial_{xx}+\partial_{zz}$ is the Laplacian differential operator , $\left(u,w\right)$ are the components 2\textendash dimensional velocity vector field, $P$ is the pressure anomaly field, $b$ is the buoyancy field, $N$ is the Br�nt-V�is�l� frequency (considered constant here), $\nu$ is the kinematic viscosity and $\alpha$ is a relaxation time toward an equilibrium stratification for the buoyancy. The first equation will be referred to as the zonal momentum equation, the second one as the vertical momentum equation, the third one as the buoyancy equation and the last one as the incompressibility condition. The buoyancy is defined from the mass density $\rho$ through $b=g\frac{\left(\rho_{c}-\rho\right)}{\rho_{c}}-N^{2}z$ where $\rho_{c}$ is a reference value for $\rho$.

\subsubsection*{Boundary conditions}

The domain is periodic in the zonal direction. For any field $\phi\left(x,z\right)$, we have $\phi\left(x+nL\right)=\phi\left(x\right)$ $\forall\left(x,n\right)\in\mathbb{R}\times\mathbb{Z}$. $L$ is the zonal period of the domain. At the boundary at the bottom of the domain will be referred to as topography. Its equation is given by $z\left(x,t\right)=h_{b}\left(x,t\right)$ with $\int_{0}^{L}h_{b}\left(x,t\right)\mathrm{d}x=0$. 

If $\nu=0$, we consider simply the impermeability condition at the bottom boundary. It amounts to fixe the normal velocity at the boundary to be equal to the normal velocity of the topography :
\begin{equation}
	\left(\mathbf{u}\left(x,z=h_{b}\left(x,t\right)\right)-\mathbf{v}_{h_{b}}\left(x,t\right)\right)\cdot\mathbf{n}_{h_{b}}\left(x,t\right)=0\;\forall x,t
\end{equation}

If $\nu\neq0$, we can either consider a no\textendash slip or a free\textendash slip boundary condition :
\begin{itemize}
	\item No\textendash slip 
	\begin{equation}
		\mathbf{u}\left(x,z=h_{b}\left(x,t\right),t\right)=\mathbf{v}_{h_{b}}\left(x,t\right)\quad\forall x,t
	\end{equation}
	\item Free\textendash slip 
	\begin{equation}
		\begin{cases}
			\left(\mathbf{u}\left(x,z=h_{b}\left(x,t\right),t\right)-\mathbf{v}_{h_{b}}\left(x,t\right)\right)\cdot\mathbf{n}_{h_{b}}\left(x,t\right) 															& =0\\
			\mathbf{n}_{h_{b}}^{\bot}\left(x,t\right)\cdot\left(\overline{\overline{\sigma}}\left(\mathbf{u}\right)\left(x,z=h_{b}\left(x,t\right)\right)\left[\mathbf{n}_{h_{b}}\left(x,t\right)\right]\right) & =0
		\end{cases}\;\forall x,t,
	\end{equation}
	the first equation is the impermeability condition, the second one is the condition of vanishing tangential constraint. $\overline{\overline{\sigma}}$ is the constraint tensor.
\end{itemize}

\subsection{Adimensionalized equations}

Let us consider the parameters of the problem$\left\{ h_{b},N,\omega,k,\nu,\alpha\right\}$ where\textbf{ $h_{b}$, $k$} and $\omega$ are typical values of the amplitude of the topography, the horizontal wave number and the angular frequency of the topography. We denote $c=\omega/k$ the associated phase\textendash speed. We can define three independent dimensionless parameters :
\begin{equation}
	\begin{cases}
		\text{Reynold number :} 			& \mathrm{Re}=\frac{c}{\nu k}\\
		\text{Froude number : } 			& \mathrm{Fr}=\frac{\omega}{N}\\
		\text{Newton number :} 				& \mathrm{\beta=\frac{\alpha}{\omega}}\\
		\text{Topographic aspect ratio :} 	& \epsilon=h_{b}k
	\end{cases}.
\end{equation}
The adimensionalized variables writes :
\begin{equation}
	\begin{cases}
		\left(\tilde{x},\tilde{z}\right) 	& =k\left(x,z\right)\\
		\tilde{t} 							& =\omega t\\
		\left(\tilde{u},\tilde{w}\right) 	& =\left(u,w\right)/c\\
		\tilde{b} 							& =kb/N^{2}\\
		\tilde{P} 							& =P/c^{2}
	\end{cases}
\end{equation}
This leads to, if we drop the tilde :
\begin{equation}
	\begin{cases}
		\partial_{t}u+u\partial_{x}u+w\partial_{z}u 	& =-\partial_{x}P+\frac{1}{\mathrm{Re}}\Delta u\\
		\partial_{t}w+u\partial_{x}w+w\partial_{z}w 	& =-\partial_{z}P+\frac{1}{\mathrm{Fr}^{2}}b+\frac{1}{\mathrm{Re}}\Delta w\\
		\partial_{t}b+u\partial_{x}b+w\partial_{z}b+w 	& =-\beta b\\
		\partial_{x}u+\partial_{z}w 					& =0
	\end{cases}\label{eq:GlobalEQ_NonLin}
\end{equation}
Let us write down the bottom boundary conditions :
\begin{itemize}
	\item Impermeability
		\begin{equation}
			w-v_{z,hb}=\epsilon\left(u-u_{x,h_{B}}\right)\partial_{x}h_{b}|_{z=\epsilon h_{b}}
		\end{equation}
	\item Viscous boundary conditions :
\end{itemize}

\paragraph{Free-slip : No tangential constraint }
\begin{equation}
	\left(\partial_{z}u+\partial_{x}w\right)\frac{\left(1-\left(\epsilon\partial_{x}h_{b}\right)^{2}\right)}{2}-\epsilon\partial_{x}h_{b}\left(\partial_{x}u-\partial_{z}w\right)=0|_{z=\epsilon h_{b}}
\end{equation}

\paragraph{No-slip : Cancelation of the velocity}
\begin{equation}
	\mathbf{u}|_{z=\epsilon h_{b}}=\mathbf{v}_{h_{b}}
\end{equation}

\section{Zonally symmetric wave\-- mean flow interaction}

Let us introduce the zonal average operator by 
\begin{equation}
	\overline{\phi}\left(z,t\right)=\frac{1}{2\pi}\int_{0}^{2\pi}\mathrm{d}x\;\phi\left(x,z,t\right),
\end{equation}
where $\phi$ is any field. We define the waves-mean flow decomposition by 
\begin{equation}
	\left(u,w,b,P\right)=\left(\overline{u},\overline{w},\overline{b},\overline{P}\right)+\left(u',w',b',P'\right),
\end{equation}
the fields $\left(\overline{u},\overline{w},\overline{b},\overline{P}\right)$ will be referred to as the mean\textendash flow and the fields $\left(u',w',b',P'\right)$ as the waves. 

We can notice that if there is no Newtonian cooling (i.e. $\beta=0$), then the mass distribution is conserved and, thus, the mass field as to be a rearrangement of the initial mass field. If we assume this initial field to by linear then $\overline{b}+b'=N^{2}\zeta\left(x,z,t\right)$ where $\zeta$ is a vertical displacement field. From incompressibility, we have $\overline{\zeta}=0$. This means that the mean buoyancy is zero at any time. For a non linear initial stratification $S_{bg}\left(z\right)$, $\overline{b}$ will be given by a series $\sum_{k=2}^{\infty}S_{bg}^{\left(k\right)}\left(z\right)\overline{\zeta^{k}}/k!$ which is bounded by the amplitude of the fluctuations. 

The coupled waves-mean flow equations are given by simply applying the zonal average to\textbf{ (\ref{eq:GlobalEQ_NonLin}) }and using the zonal periodicity :
\begin{equation}
	\begin{cases}
		\partial_{t}\overline{u}-\frac{1}{\mathrm{Re}}\partial_{zz}\overline{u} & =-\partial_{z}\overline{w'u'}\\
		\overline{P} 															& =\overline{w'^{2}}\\
		\overline{w} 															& =0\\
		\partial_{t}\overline{b} 												& =-\partial_{z}\overline{b'w'}-\beta\overline{b}
	\end{cases}
\end{equation}
The second equation is simply a diagnostic equation giving the mean pressure field. Also the fourth equation has to be satisfied by the diagnostic expression of $\overline{b}$ given above. This means that $\partial_{z}\overline{b'w'}$ has to be a time partial\textendash derivative of a bounded quantity. We focus ourselves on the mean momentum equation carrying all the interesting physics :
\begin{equation}
	\boxed{\partial_{t}\overline{u}-\frac{1}{\mathrm{Re}}\partial_{zz}\overline{u}=-\partial_{z}\overline{u'w'}}.
\end{equation}
The important term is the Reynold stress $\partial_{z}\overline{u'w'}$ in the zonal mean flow equation. It is computed from the fluctuations, i.e. the waves in the limit of small amplitude. Subtracting the mean\textendash flow equations to \textbf{(\ref{eq:GlobalEQ_NonLin})}, we obtain the equations for the fluctuations
\begin{equation}
	\begin{cases}
		\partial_{t}u'+\overline{u}\partial_{x}u'+w'\partial_{z}\overline{u}+\mathcal{B}\left(\mathbf{u}',u'\right)	& =-\partial_{x}P'+\frac{1}{\mathrm{Re}}\Delta u'\\
		\partial_{t}w'+\overline{u}\partial_{x}w'+\mathcal{B}\left(\mathbf{u}',w'\right) 							& =-\partial_{z}P'+\frac{1}{\mathrm{Fr}^{2}}b'+\frac{1}{\mathrm{Re}}\Delta w'\\
		\partial_{x}u'+\partial_{z}w' 																				& =0\\
		\partial_{t}b'+\overline{u}\partial_{x}b''+w'+\mathcal{B}\left(\mathbf{u}',b'\right)						& =-\beta b'
	\end{cases}
\end{equation}
where $\mathcal{B}\left(\mathbf{u}',\phi'\right)=u'\partial_{x}\phi'+w'\partial_{z}\phi'-\overline{u'\partial_{x}\phi'+w'\partial_{z}\phi}$ is a bilinear operator with zero zonal average.

In principle, we perform the same flow decomposition at the boundaries in order to get the boundary condition for wave and mean part but this formulation make sense only in a far-field approximation when the domain geometry is asymptotically flat. If one want to consider a proper definition of the mean flow close to the topography, one would have to consider the Lagrangian-mean theory.

\section{Internal waves within a weakly sheared mean-flow}

We consider here a single topographic mode $k=\pm1$ and a single phase speed $c=\pm1$. We consider the limit a small topographic aspect ratio (i.e. $\epsilon\ll1$). In this limit we can consider an asymptotically square domain with a bottom boundary condition expressed at $z=0$ in terms of a series of power of $\epsilon$:
\begin{itemize}
	\item Impermeability
	\begin{equation}
		\sum_{n=2}^{\infty}\epsilon^{n}h_{b}^{n-1}\left(h_{b}\partial_{z}^{n}w-\partial_{x}h_{b}\partial_{z}^{n-1}u\right)+w-v_{z,h_{b}}+\epsilon\left(h_{b}\partial_{z}w-\left(u-u_{x,h_{B}}\right)\partial_{x}h_{b}|_{z=\epsilon h_{b}}\right)=0.
	\end{equation}
	\item Viscous boundary conditions :
\end{itemize}

\paragraph{Free-slip : No tangential constraint}
\begin{equation}
	\sum_{n=2}^{\infty}\epsilon^{n}\left(\frac{h_{b}^{n}}{2}\left(\partial_{z}^{n+1}u+\partial_{x}\partial_{z}^{n}w\right)-\frac{\partial_{x}\left(h_{b}^{n-1}\right)\partial_{x}h_{b}}{2\left(n-1\right)}\left(\partial_{z}^{n-1}u+\partial_{x}\partial_{z}^{n-2}w\right)-\frac{\partial_{x}\left(h_{b}^{n}\right)}{n}\left(\partial_{x}\partial_{z}^{n-1}u-\partial_{z}^{n+1}w\right)\right)+\epsilon\left(\frac{h_{b}}{2}\left(\partial_{zz}u+\partial_{zx}w\right)-\partial_{x}h_{b}\left(\partial_{x}u-\partial_{z}w\right)\right)+\frac{1}{2}\left(\partial_{z}u+\partial_{x}w\right)=0.
\end{equation}


\paragraph{No-slip : Cancelation of the velocity}
\begin{equation}
	\sum_{n=1}^{\infty}\epsilon^{n}h_{b}^{n}\partial_{z}^{n}\mathbf{u}=\mathbf{v}_{h_{b}}.
\end{equation}


Let us introduce a small parameter $\epsilon^{2}\ll a\ll1$. We will see later how it scales with $\epsilon$. Let us introduce a slow variable $Z=az$. We now assume that the mean flow $\overline{u}$ depend only on $Z$. This mean that for the wave field, the mean flow appears weakly sheared and frozen in time.

Let us introduce the following WKB expansion of a stationary solution of the wave-equation
\begin{equation}
	\left[
	\begin{array}{c}
		u'\\
		w'\\
		b'\\
		p'
	\end{array}
	\right] = \epsilon\sum_{j=0}^{+\infty}a^{j}\mathbb{R}\mathrm{e}\left\{ \left[
	\begin{array}{c}
		\tilde{u}{}_{j}\left(Z\right)\\
		\tilde{w}{}_{j}\left(Z\right)\\
		\tilde{b}{}_{j}\left(Z\right)\\
		\tilde{p}_{j}\left(Z\right)
	\end{array}
	\right]\exp\left(ik\left(ct-x\right)-\frac{i\Phi\left(Z\right)}{a}\right)\right\} .
\end{equation}
Injecting this expansion into the previous equation and collecting the leading order in $a$ leads :
\begin{equation}
	\epsilon\boldsymbol{\mathrm{M}}\left[
	\begin{array}{c}
		\tilde{u}{}_{0}\\
		\tilde{w}{}_{0}\\
		\tilde{b}{}_{0}\\
		\tilde{p}{}_{0}
	\end{array}
	\right]+\epsilon a\left(\boldsymbol{\mathrm{M}}\left[
	\begin{array}{c}
		\tilde{u}{}_{1}\\
		\tilde{w}{}_{1}\\
		\tilde{b}{}_{1}\\
		\tilde{p}{}_{1}
	\end{array}
	\right]+\left[
	\begin{array}{c}
		\tilde{w}{}_{0}\partial_{Z}\overline{u}+\frac{i}{\mathrm{Re}}\left(\tilde{u}_{0}\partial_{Z}m+2m\partial_{Z}\tilde{u}_{0}\right)\\
		\partial_{Z}\tilde{P}{}_{0}+\frac{i}{\mathrm{Re}}\left(\tilde{w}_{0}\partial_{Z}m+2m\partial_{Z}\tilde{w}_{0}\right)\\
		0\\
		\partial_{Z}\tilde{w}{}_{0}
	\end{array}
	\right]\right)+o\left(\epsilon a\right)=0
\end{equation}
with 
\begin{equation}
	\boldsymbol{\mathrm{M}}=\left[
	\begin{array}{cccc}
		ik\left(c-\overline{u}\right)\left(1-i\frac{k^{2}+m^{2}}{\mathrm{Re}k\left(c-\overline{u}\right)}\right) & 0 & 0 & -ik\\
		0 & ik\left(c-\overline{u}\right)\left(1-i\frac{k^{2}+m^{2}}{\mathrm{Re}k\left(c-\overline{u}\right)}\right) & -\frac{1}{\mathrm{Fr}^{2}} & -im\\
		0 & 1 & ik\left(c-\overline{u}\right)\left(1-i\frac{\beta}{k\left(c-\overline{u}\right)}\right) & 0\\
		-ik & -im & 0 & 0
	\end{array}
	\right]\qquad\text{ and }m=\partial_{Z}\phi
\end{equation}
This implies 
\begin{equation}
	\det\boldsymbol{\mathrm{M}}=0\implies\mathrm{Fr}^{2}\left(c-\overline{u}\right)^{2}\left(1-i\frac{\left(k^{2}+m^{2}\right)}{\mathrm{Re}k\left(c-\overline{u}\right)}\right)\left(1-i\frac{\beta}{k\left(c-\overline{u}\right)}\right)\left(k^{2}+m^{2}\right)=1\qquad\text{ and }\qquad\left[
	\begin{array}{c}
		\tilde{u}{}_{0}\\
		\tilde{w}{}_{0}\\
		\tilde{b}{}_{0}\\
		\tilde{p}{}_{0}
	\end{array}
	\right]=\phi_{0}\left(Z\right)\boldsymbol{\mathcal{P}}\left(Z\right)
\end{equation}
where $\tilde{\phi}_{0}$ measures the global amplitude of the wave and $\boldsymbol{\mathcal{P}}$ is the polarization of the wave given by 
\begin{equation}
	\boldsymbol{\mathcal{P}}=\left[
		\begin{array}{c}
			c-\overline{u}\\
			-{\displaystyle \frac{k}{m}}\left(c-\overline{u}\right)\\
			{\displaystyle -\frac{i}{m\left(1-i\frac{\beta}{k\left(c-\overline{u}\right)}\right)}}\\
			\left(c-\overline{u}\right)^{2}\left(1-i\frac{\left(k^{2}+m^{2}\right)}{\mathrm{Re}k\left(c-\overline{u}\right)}\right)
		\end{array}
	\right].
\end{equation}
We now project the order one terms on
\begin{equation}
	\left[
	\begin{array}{c}
		c-\overline{u}\\
		-{\displaystyle \frac{k}{m}}\left(c-\overline{u}\right)\\
		{\displaystyle \frac{i}{\mathrm{Fr}^{2}m\left(1-i\frac{\beta}{k\left(c-\overline{u}\right)}\right)}}\\
		\left(c-\overline{u}\right)^{2}\left(1-i\frac{\left(k^{2}+m^{2}\right)}{\mathrm{Re}k\left(c-\overline{u}\right)}\right)
	\end{array}
	\right].
\end{equation}
This leads toward a messy equation for the wave\-- amplitude $\phi_{0}\left(Z\right)$ :
\begin{equation}
	\frac{1}{\phi_{0}\left(c-\overline{u}\right)}\partial_{Z}\left(\phi_{0}\left(c-\overline{u}\right)\right)=\frac{2\left(k^{2}-m^{2}\right)+ik\mathrm{Re}\left(c-\overline{u}\right)}{4m^{2}\left(2\left(k^{2}+m^{2}\right)+ik\mathrm{Re}\left(c-\overline{u}\right)\right)}\partial_{Z}m^{2}
\end{equation}


\subsection{Case $\mathrm{Re=\infty}$ and $\beta>0$:}

We have 
\begin{equation}
	\mathrm{Fr}^{2}\left(c-\overline{u}\right)^{2}\left(1-i\frac{\beta}{k\left(c-\overline{u}\right)}\right)\left(k^{2}+m^{2}\right)=1.
\end{equation}
This is an 2nd-order equation. There is only one propagative branch for $m$ . In the limit of $\beta\ll1$, we have :
\begin{equation}
	m=\pm\left(\underbrace{\sqrt{\frac{1}{\left(c-\overline{u}\right)^{2}\mathrm{Fr}^{2}}-k^{2}}}_{m_{0}}+i\frac{\beta}{2\left(c-\overline{u}\right)^{3}\mathrm{Fr}^{2}\sqrt{\frac{1}{\left(c-\overline{u}\right)^{2}\mathrm{Fr}}-k^{2}}}\right).
\end{equation}
In the limit $\beta\ll1$, we also have :

\begin{equation}
	\partial_{Z}\left(\frac{\phi_{0}^{2}\left(c-\overline{u}\right)^{2}}{\sqrt{\frac{1}{\left(c-\overline{u}\right)^{2}\mathrm{Fr}^{2}}-k^{2}}}\right)=0.
\end{equation}
The vertical group velocity is defined by : 
\begin{equation}
	w_{g}\equiv\frac{k}{\partial_{c}\mathbb{Re}\left[m\right]}\protect\underset{\beta\to0}{\sim}\mp\frac{km_{0}}{k^{2}+m_{0}^{2}}\left(c-\overline{u}\right)
\end{equation}


\subsection{Case $\beta=0$ and $\mathrm{Re}<\infty$ :}

We have 
\begin{equation}
	\mathrm{Fr}^{2}\left(c-\overline{u}\right)^{2}\left(1-i\frac{\left(k^{2}+m^{2}\right)}{\mathrm{Re}k\left(c-\overline{u}\right)}\right)\left(k^{2}+m^{2}\right)=1.
\end{equation}
This an 4th\-- order equation. There is two branches for $m$. A propagative one and a boundary layer one. In the limit of $\mathrm{Re}\gg1$, we have :
\begin{itemize}
	\item if $k^{2}\left(c-\overline{u}\right)^{2}\mathrm{Fr}^{2}<1$:
	\begin{align*}
		m_{\mathrm{Prop}} 	& =\pm\left(\sqrt{\frac{1}{\left(c-\overline{u}\right)^{2}\mathrm{Fr^{2}}}-k^{2}}+\frac{i}{2k\left(c-\overline{u}\right)^{3}\mathrm{Fr}^{2}\mathrm{Re}\sqrt{\frac{1}{\left(c-\overline{u}\right)^{2}\mathrm{Fr}^{2}}-k^{2}}}\right)+o\left(\mathrm{Re}^{-1}\right)\\
		m_{\mathrm{bl}} 	& =\pm\left(1-\mathrm{sign}\left(k\right)i\right)\sqrt{\frac{\left|c-\overline{u}\right|\left|k\right|\mathrm{Re}}{2}}+o\left(\sqrt{\mathrm{Re}}\right)
	\end{align*}
	\begin{align*}
		m_{\mathrm{Prop}}^{2} 	& =\frac{1}{\left(c-\overline{u}\right)^{2}\mathrm{Fr}^{2}}-k^{2}+\frac{i}{\left(c-\overline{u}\right)^{5}\mathrm{Fr}^{4}k\mathrm{Re}}+o\left(\mathrm{Re}^{-1}\right)\\
		m_{\mathrm{bl}}^{2} 	& =-i\left(c-\overline{u}\right)k\mathrm{Re}+o\left(\mathrm{Re}\right)
	\end{align*}
	Then for $m_{\mathrm{Prop}}$, we have at leading order in $\mathrm{Re}$:
	\begin{equation}
		\partial_{Z}\left(\frac{\phi_{0,\mathrm{Prop}}^{2}\left(c-\overline{u}\right)^{2}}{\sqrt{\frac{1}{\left(c-\overline{u}\right)^{2}\mathrm{Fr}^{2}}-k^{2}}}\right)=0
	\end{equation}
	and for $m_{\mathrm{bl}}$, we have at leading order in $\mathrm{Re}$:
	\begin{equation}
		\partial_{Z}\left(\phi_{0,\mathrm{bl}}m_{\mathrm{bl}}\right)=0
	\end{equation}
	\item if $k^{2}\left(c-\overline{u}\right)^{2}\mathrm{Fr}^{2}>1$:
	\begin{align*}
		m_{\mathrm{Prop}} 	& =\pm\left(i\sqrt{k^{2}-\frac{1}{\left(c-\overline{u}\right)^{2}\mathrm{Fr^{2}}}}+\frac{1}{2k\left(c-\overline{u}\right)^{3}\mathrm{Fr}^{2}\mathrm{Re}\sqrt{k^{2}-\frac{1}{\left(c-\overline{u}\right)^{2}\mathrm{Fr^{2}}}}}\right)+o\left(\mathrm{Re}^{-1}\right)\\
		m_{\mathrm{bl}} 	& =\pm\left(1-\mathrm{sign}\left(k\left(c-\overline{u}\right)\right)i\right)\sqrt{\frac{\left|c-\overline{u}\right|\left|k\right|\mathrm{Re}}{2}}+o\left(\sqrt{\mathrm{Re}}\right)
	\end{align*}
	\begin{align*}
		m_{\mathrm{Prop}}^{2}	& =\frac{1}{\left(c-\overline{u}\right)^{2}\mathrm{Fr}^{2}}-k^{2}+\frac{i}{\left(c-\overline{u}\right)^{5}\mathrm{Fr}^{4}k\mathrm{Re}}+o\left(\mathrm{Re}^{-1}\right)\\
		m_{\mathrm{bl}}^{2} 	& =-i\left(c-\overline{u}\right)k\mathrm{Re}+o\left(\mathrm{Re}\right)
	\end{align*}
	Then for $m_{\mathrm{Prop}}$, we have at leading order in $\mathrm{Re}$:
	\begin{equation}
		\partial_{Z}\left(\frac{\phi_{0,\mathrm{Prop}}^{2}\left(c-\overline{u}\right)^{2}}{\sqrt{k^{2}-\frac{1}{\left(c-\overline{u}\right)^{2}\mathrm{Fr}^{2}}}}\right)=0
	\end{equation}
	and for $m_{\mathrm{bl}}$, we have at leading order in $\mathrm{Re}$:
	\begin{equation}
		\partial_{Z}\left(\phi_{0,\mathrm{bl}}m_{\mathrm{bl}}\right)=0
	\end{equation}
\end{itemize}
\bibliographystyle{plain}
\bibliography{Bibliographie}

\end{document}
