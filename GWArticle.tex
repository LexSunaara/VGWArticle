\documentclass[english]{article}
\usepackage[T1]{fontenc}
\usepackage[latin9]{inputenc}
\usepackage{amsmath}
\usepackage{amssymb}
\usepackage[margin=2cm]{geometry}

\usepackage{babel}
\begin{document}

\title{Generation of viscous internal gravity waves by a moving boundary}
\author{A. Renaud and A. Venaille}
\maketitle

\section{Introduction}

Internal gravity waves play a major role in geophysical fluid dynamics. The underlying reason of their increasing importance comes from their primary contribution to the mixing in the ocean (see \cite{Wunsch2004}) and in the redistribution of energy and momentum in the atmosphere (see \cite{Fritts1984}). 

They have been extensively studied theoretically, numerically and experimentally. The well understood inviscid theory (see \cite{Bell1975,Muraschko2015}) is relevant to describe most of oceanic and atmospheric internal waves related phenomena. However, the large difference in spatial and time scaling between geophysical internal waves and there laboratory analogues makes it dodgy to ignore the viscous effects in the latter context. In numerical simulations with high enough resolution to capture the internal waves dynamics, the unresolved smaller scales are often parametrised by effective viscosities several order of magnitude higher than the actual geophysical ones (\textbf{REF ?}). The theory of the quasi\--biennial oscillation occurring in the equatorial stratosphere needs a dissipative process to attenuate the waves beams coming from tropical troposphere (see \cite{Lindzen1968,Plumb1977}). In the atmospheric context, this dissipation comes mainly from the Newtonian cooling parametrised by a drag force in the buoyancy equation (see e.g. \cite{Grisouard2012}). However, in the laboratory analogue performed by \cite{Plumb1978} the dominant dissipation process comes from the viscosity. The main difference between these two features is that the latter one involves viscous boundary layers to match the propagating waves with the boundary condition at level of the waves generator. At a quasi\--linear level those boundary layers can involve an important streaming close to the generator.

The way internal gravity waves are generated in experiments matters a lot if we take into account the viscous effects associated with the no\--slip boundary condition. A first class of experiments consists in dragging an undulated boundary within the fluid (see \cite{Long1955} for pioneer work and \cite{Dossmann2016} for more recent work). A second class of experiments uses wave\--beam generator introduced by \cite{Gostiaux2006} which allows for the generation of progressive wave\--beams.

Plan du papier :
\begin{itemize}
	\item Section 1 : Modele, equation primitive, approximation de boussinesq
	\item Section 2 : Zonally symetric waves and mean\--flow interaction
	\item Section 3 : WKB asymptotics and wave properties
	\item Section 4 : Boundary condition and reynold stress
	\item Section 5 : Applications : Lee waves, QBO Plumb
	\item Section 6 : Simulations
	\item Section 7 : Discussion/Conclusion
\end{itemize}





\begin{table}
	\begin{tabular}{|c|c|c|c|c|c|c|}
	\hline 
	\textbf{Studies} 	& $\mathrm{Re}$	& $\mathrm{Fr}$	& $\epsilon$	& $\mathrm{Pr}$	& Generation			& Type\tabularnewline
	\hline 
	\hline 
	\cite{Baines1985} 	& $100-1000$ 	& $0.1$ 		& $0.1$ 		& $1000$ 		& ridge drag 			& Experimental\tabularnewline
	\hline 
	\cite{Dossmann2016} & $400-87000$ 	& $0.1-34$ 		& $0.1-1.$ 		& $1000$ 		& ridge drag 			& Experimental\tabularnewline
	\hline 	
	\cite{Peacock2009} 	& $73$ 			& $0.82$ 		& $0.4$ 		& $1000$ 		& oscillating cams 		& Experimental\tabularnewline
	\hline 
	\cite{Plumb1978} 	& $1500-3000$ 	& $0.02-0.04$ 	& $0.05$ 		& $1000$ 		& oscillating membranes & Experimental\tabularnewline
	\hline 
	\cite{Semin2016} 	& $650$ 		& $0.1 $		& $0.06$ 		& $1000$ 		& oscillating membranes & Experimental\tabularnewline
	\hline 
\end{tabular}

\caption{Referencing of some characteristic dimensionless number in internal gravity wave experiments.\label{tab:Referencing-of-some} }
\end{table}


\section{The Boussinesq model}

We consider a stratified fluid within a  2\--dimensional domain periodic in the horizontal $x$\--direction (this direction will be referred to as ``zonal'' in the remaining of the paper) and semi-infinite in the vertical $z$\--direction. We consider the incompressible Boussinesq model. Two types of dissipative mechanism will be considered: viscosity and Newtonian cooling. The equation of state is linear. The bottom boundary is an undulating line located on average at $z=0$. We are interested in the dynamics of the buoyancy anomaly defined by $b=g\frac{\left(\rho_{c}-\rho\right)}{\rho_{c}}-\int^{z}N^{2}$ where $\rho_{c}$ is a reference value for the density field $\rho$ and $N$ is the Br�nt-V�is�l� frequency (height\--dependent in general and always positive such that the rest state, $b=0$, is stable). The parameters of the problem are the characteristic amplitude, wave number and angular momentum of the bottom boundary undulation $\left(h_{b},k,\omega\right)$, the kinematic viscosity $\nu$ and the bottom value of the Br�nt\--V�is�l� frequency $N_{0}=N\left(z=0\right)$. We denote $c=\omega/k$ the associated phase\--speed. We can define three independent dimensionless parameters:
\begin{equation}
	\begin{cases}
		\text{Reynold number :} 				& \mathrm{Re}=\frac{c}{\nu k}\\
		\text{Froude number : } 				& \mathrm{Fr}=\frac{\omega}{N_{0}}\\
		\text{Bottom boundary aspect ratio :}	& \epsilon=h_{b}k
	\end{cases}.
\end{equation}

\subsection{Equations}

The flow obeys the primitive equations in the Boussinesq approximation. In the adimensionalized form, they writes:
\begin{equation}
	\begin{cases}
		\partial_{t}u+u\partial_{x}u+w\partial_{z}u 	& =-\partial_{x}P+\frac{1}{\mathrm{Re}}\Delta u\\
		\partial_{t}w+u\partial_{x}w+w\partial_{z}w 	& =-\partial_{z}P+\frac{1}{\mathrm{Fr}^{2}}b+\frac{1}{\mathrm{Re}}\Delta w\\
		\partial_{t}b+u\partial_{x}b+w\partial_{z}b+n^{2}\left(z\right)w 	& =-\alpha b\\
		\partial_{x}u+\partial_{z}w 					& =0
	\end{cases}\label{eq:GlobalEQ_NonLin}
\end{equation}
where  $\Delta=\partial_{xx}+\partial_{zz}$ is the Laplacian differential operator , $\left(u,w\right)$ are the components of the 2\--dimensional velocity vector field, $P$ is the pressure anomaly field, $b$ is the buoyancy anomaly field, $n$ is the rescaled Br�nt-V�is�l� frequency field ($n\left(0\right)=1$ and $n\left(z>0\right)=1$ if the stratification is linear) and $\alpha$ is a parameter parameterizing the relaxation of the buoyancy anomaly toward the rest state.  The first equation will be referred to as the zonal momentum equation, the second one as the vertical momentum equation, the third one as the buoyancy equation and the last one as the incompressibility condition. 

\subsubsection*{Boundary conditions}

The domain is $2\pi$\--periodic in the zonal direction. The equation of the bottom boundary is given by $z\left(x,t\right)=\epsilon h_{b}\left(x,t\right)$ with $\int_{0}^{2\pi}h_{b}\left(x,t\right)\mathrm{d}x=0$. All the following equation for the boundary conditions have to be evaluated in $z= \epsilon h_{b}\left(x,t\right)$.

If $\mathrm{Re}=\infty$, we consider an impermeability condition at the bottom boundary. It amounts to fix the normal velocity at the boundary to be equal to its normal velocity:
\begin{equation}
	w-w_{h_{b}}-\epsilon\left(u-u_{h_{b}}\right)\partial_{x}h_{b} 	= 0 
\end{equation}
where $\left(u_{h_{b}},w_{h_{b}}\right)$ are the component of the velocity of the bottom boundary.

If $\mathrm{Re}<\infty$, we will either consider a no\--slip or a free\--slip boundary condition :
\begin{itemize}
	\item No\--slip 
	\begin{equation}
			\left(u,w\right) = \left(u_{h_{b}},w_{h_{b}}\right)
	\end{equation}
	\item Free\textendash slip 
	\begin{equation}
		\begin{cases}
			w-w_{h_{b}}-\epsilon\left(u-u_{h_{b}}\right)\partial_{x}h_{b}																													& 	= 0 	\\
			\left(\partial_{z}u+\partial_{x}w\right)\left(1-\left(\epsilon\partial_{x}h_{b}\right)^{2}\right)-2\epsilon\partial_{x}h_{b}\left(\partial_{x}u-\partial_{z}w\right)			& 	= 0
		\end{cases}
	\end{equation}
	the first equation is the impermeability condition, the second one is the condition of vanishing tangential constraint.
\end{itemize}

We consider the limit of a small topographic aspect ratio (i.e. $\epsilon\ll1$). In this limit we can consider the bottom boundary to be asymptotically flat. The bottom boundary condition is expressed at $z=0$ in terms of a series of $\epsilon$. At leading order in $\epsilon$ in writes:
\begin{itemize}
	\item Inviscid impermeability condition
	\begin{equation}
		w-w_{h_{b}}+\epsilon\left(h_{b}\partial_{z}w-(u-u_{h_{b}}\right)\partial_{x}h_{b})=o\left(\epsilon\right)\label{eq:BC_1}
	\end{equation}
	\item Viscous boundary conditions :\\
	\textbf{Free-slip}
	\begin{equation}
		\partial_{z}u+\partial_{x}w+\epsilon\left(h_{b}\left(\partial_{zz}u+\partial_{xz}w \right)-2\partial_{x}h_{b}\left(\partial_{x}u-\partial_{z}w\right)\right)=o\left(\epsilon\right)\label{eq:BC_2}
	\end{equation} \\
	\textbf{No-slip}
	\begin{equation}
		\left(u,w\right)-\left(u_{h_{b}},w_{h_{b}}\right)+\epsilon h_{b}\left(\partial_{z}u,\partial_{z}w\right)=o\left(\epsilon\right)\label{eq:BC_3}
	\end{equation}
\end{itemize}

\section{Zonally symmetric wave\--mean flow interaction \label{sec:Zonally}}

Let us introduce the zonal average operator:
\begin{equation}
	\overline{\phi}\left(z,t\right)=\frac{1}{2\pi}\int_{0}^{2\pi}\mathrm{d}x\;\phi\left(x,z,t\right),
\end{equation}
where $\phi$ is any field. We define the waves-mean flow decomposition by 
\begin{equation}
	\left(u,w,b,P\right)=\left(\overline{u},\overline{w},\overline{b},\overline{P}\right)+\left(u',w',b',P'\right),
\end{equation}
the fields $\left(\overline{u},\overline{w},\overline{b},\overline{P}\right)$ will be referred to as the mean\textendash flow and the fields $\left(u',w',b',P'\right)$ as the waves. 

We can notice that if there is no Newtonian cooling (i.e. $\alpha=0$ in Eq. (\ref{eq:GlobalEQ_NonLin})), any integral over the domain of the total buoyancy $b+\int^{z}n^{2}$ is conserved by the dynamics. It implies that the total buoyancy field has to be a rearrangement of its initial condition. This means that the mean buoyancy field is of order two in terms of the wave amplitude. In the following we will consider small amplitude waves. We, hereby, ignore the equation of evolution of the mean buoyancy field $\overline{b}$ and we make the approximation $\overline{b}=0$ in the following. The averaging of the continuity equation in (\ref{eq:GlobalEQ_NonLin}) leads to $\overline{w}=0$. Consequently, the averaging of the vertical momentum equation simply leads to a diagnostic equation for the mean pressure field $\overline{P}$. The equation of interest is the averaged zonal momentum equation : 
\begin{equation}
	\partial_{t}\overline{u}-\frac{1}{\mathrm{Re}}\partial_{zz}\overline{u}=-\partial_{z}\overline{u'w'}.
\end{equation}
The important term is the Reynold stress $\partial_{z}\overline{u'w'}$. The wave equation need to be solved to compute it. Subtracting the mean\--flow equations to (\ref{eq:GlobalEQ_NonLin}), we obtain the fully non\--linear wave equation:
\begin{equation}
	\begin{cases}
		\partial_{t}u'+\overline{u}\partial_{x}u'+w'\partial_{z}\overline{u}+\mathcal{B}\left(\mathbf{u}',u'\right)	& =-\partial_{x}P'+\frac{1}{\mathrm{Re}}\Delta u'\\
		\partial_{t}w'+\overline{u}\partial_{x}w'+\mathcal{B}\left(\mathbf{u}',w'\right) 							& =-\partial_{z}P'+\frac{1}{\mathrm{Fr}^{2}}b'+\frac{1}{\mathrm{Re}}\Delta w'\\
		\partial_{x}u'+\partial_{z}w' 																				& =0\\
		\partial_{t}b'+\overline{u}\partial_{x}b''+n^{2}\left(z\right)w'+\mathcal{B}\left(\mathbf{u}',b'\right)						& =-\alpha b'
	\end{cases}\label{eq:GlobalFluct}
\end{equation}
where $\mathcal{B}\left(\mathbf{u}',\phi'\right)=u'\partial_{x}\phi'+w'\partial_{z}\phi'-\overline{u'\partial_{x}\phi'+w'\partial_{z}\phi}$ is a bilinear operator with zero zonal average.

We perform the same flow decomposition at the boundaries in order to get the boundary condition for the wave and the mean part. The results are messy in general but there expressions are greatly simplified in a far-field approximation when the domain geometry is asymptotically flat in the limit $\epsilon\ll1$. 

\section{Internal gravity waves within a weakly sheared mean-flow}

Let us introduce a small parameter $\epsilon^{2}\ll a\ll1$. We will see later how it scales with $\epsilon$. Let us introduce a slow variable $Z=az$. We now assume that the mean flow $\overline{u}$ depends only on $Z$ in Eq. (\ref{eq:GlobalFluct}), we assume the same for $n$. For the wave field, this means that the mean flow appears weakly sheared and frozen in time. We will see later under which condition this hypothesis holds.

Let us introduce the WKB expansion of a stationary solution of the wave equation (\ref{eq:GlobalFluct}) :
\begin{equation}
	\left[
	\begin{array}{c}
		u'\\
		w'\\
		b'\\
		p'
	\end{array}
	\right] = \epsilon\mathbb{R}\mathrm{e}\left\{\sum_{j=0}^{+\infty}a^{j} \left[
	\begin{array}{c}
		\tilde{u}{}_{j}\left(Z\right)\\
		\tilde{w}{}_{j}\left(Z\right)\\
		\tilde{b}{}_{j}\left(Z\right)\\
		\tilde{p}_{j}\left(Z\right)
	\end{array}
	\right]\exp\left(ik\left(ct-x\right)-\frac{i\Phi\left(Z\right)}{a}\right)\right\} \label{eq:WKB_Exp}.
\end{equation}
The wave amplitude is scaling with the bottom boundary aspect ratio $\epsilon$. $\Phi\left(Z\right)$ is the phase of the wave. Injecting this expansion into the previous equation and collecting the leading order in $a$ and $\epsilon$ leads :
\begin{equation}
	\epsilon\boldsymbol{\mathrm{M}}\left[
	\begin{array}{c}
		\tilde{u}{}_{0}\\
		\tilde{w}{}_{0}\\
		\tilde{b}{}_{0}\\
		\tilde{p}{}_{0}
	\end{array}
	\right]+\epsilon a\left(\boldsymbol{\mathrm{M}}\left[
	\begin{array}{c}
		\tilde{u}{}_{1}\\
		\tilde{w}{}_{1}\\
		\tilde{b}{}_{1}\\
		\tilde{p}{}_{1}
	\end{array}
	\right]+\left[
	\begin{array}{c}
		\tilde{w}{}_{0}\partial_{Z}\overline{u}+\frac{i}{\mathrm{Re}}\left(\tilde{u}_{0}\partial_{Z}m+2m\partial_{Z}\tilde{u}_{0}\right)\\
		\partial_{Z}\tilde{P}{}_{0}+\frac{i}{\mathrm{Re}}\left(\tilde{w}_{0}\partial_{Z}m+2m\partial_{Z}\tilde{w}_{0}\right)\\
		0\\
		\partial_{Z}\tilde{w}{}_{0}
	\end{array}
	\right]\right)+o\left(\epsilon a\right)=0
\end{equation}
with 
\begin{equation}
	\boldsymbol{\mathrm{M}}=\left[
	\begin{array}{cccc}
		ik\left(c-\overline{u}\right)\left(1-i\frac{k^{2}+m^{2}}{\mathrm{Re}k\left(c-\overline{u}\right)}\right) & 0 & 0 & -ik\\
		0 & ik\left(c-\overline{u}\right)\left(1-i\frac{k^{2}+m^{2}}{\mathrm{Re}k\left(c-\overline{u}\right)}\right) & -\frac{1}{\mathrm{Fr}^{2}} & -im\\
		0 & n^{2} & ik\left(c-\overline{u}\right)\left(1-i\frac{\alpha}{k\left(c-\overline{u}\right)}\right) & 0\\
		-ik & -im & 0 & 0
	\end{array}
	\right]\qquad\text{ and }m=\partial_{Z}\phi
\end{equation}
This implies 
\begin{equation}
	\det\boldsymbol{\mathrm{M}}=0\implies\mathrm{Fr}^{2}\left(c-\overline{u}\right)^{2}\left(1-i\frac{\left(k^{2}+m^{2}\right)}{\mathrm{Re}k\left(c-\overline{u}\right)}\right)\left(1-i\frac{\alpha}{k\left(c-\overline{u}\right)}\right)\left(k^{2}+m^{2}\right)=n^{2}  \label{eq:GeneralDisp}
\end{equation}
and
\begin{equation}
	\left[
	\begin{array}{c}
		\tilde{u}{}_{0}\\
		\tilde{w}{}_{0}\\
		\tilde{b}{}_{0}\\
		\tilde{p}{}_{0}
	\end{array}
	\right]=\phi_{0}\left(Z\right)\boldsymbol{\mathcal{P}}\left(Z\right)
\end{equation}
where $\tilde{\phi}_{0}$ measures the global amplitude of the wave and $\boldsymbol{\mathcal{P}}$ is the polarization of the wave given by 
\begin{equation}
	\boldsymbol{\mathcal{P}}=\left[
		\begin{array}{c}
			c-\overline{u}\\
			-{\displaystyle \frac{k}{m}}\left(c-\overline{u}\right)\\
			{\displaystyle -\frac{i}{m\left(1-i\frac{\alpha}{k\left(c-\overline{u}\right)}\right)}}\\
			\left(c-\overline{u}\right)^{2}\left(1-i\frac{\left(k^{2}+m^{2}\right)}{\mathrm{Re}k\left(c-\overline{u}\right)}\right)
		\end{array}
	\right].
\end{equation}
 The equation (\ref{eq:GeneralDisp}) is the dispersion relation equation of the wave which as to be satisfied locally. For every value of $Z$, it gives $m\left(Z\right)$. This equation has multiple solution. Depending on the type of wave we want to study, we consider one of the value. The complete linear solution of the problem is a linear combination of all the solution satisfying the boundary conditions. 

 To get an evolution equation for $\phi_{0}$, we project the order one terms on
\begin{equation}
	\left[
	\begin{array}{c}
		c-\overline{u}\\
		-{\displaystyle \frac{k}{m}}\left(c-\overline{u}\right)\\
		{\displaystyle \frac{i n^{2}}{\mathrm{Fr}^{2}m\left(1-i\frac{\alpha}{k\left(c-\overline{u}\right)}\right)}}\\
		\left(c-\overline{u}\right)^{2}\left(1-i\frac{\left(k^{2}+m^{2}\right)}{\mathrm{Re}k\left(c-\overline{u}\right)}\right)
	\end{array}
	\right].
\end{equation}
This cancel out the order one terms of the wave and leave us with an equation for the amplitude of the zero\--th order wave $\phi_{0}\left(Z\right)$ :
\begin{equation}
	\frac{1}{\phi_{0}\left(c-\overline{u}\right)}\partial_{Z}\left(\phi_{0}\left(c-\overline{u}\right)\right)=\frac{2\left(k^{2}-m^{2}\right)+ik\mathrm{Re}\left(c-\overline{u}\right)}{4m^{2}\left(2\left(k^{2}+m^{2}\right)+ik\mathrm{Re}\left(c-\overline{u}\right)\right)}\partial_{Z}m^{2}.\label{eq:AmpEQGEN}
\end{equation}
 This last equation has to be solved for every solution $m\left(Z\right)$ of the dispersion relation (\ref{eq:GeneralDisp}).


\subsection{Dispersion relation for $\mathrm{Re=\infty}$ and $\alpha>0$:}

Without viscous dissipation, the linear wave obeys the following dispersion relation locally :
\begin{equation}
	\mathrm{Fr}^{2}\left(c-\overline{u}\right)^{2}\left(1-i\frac{\alpha}{k\left(c-\overline{u}\right)}\right)\left(k^{2}+m^{2}\right)=n^{2}.
\end{equation}
This is an 2nd-order equation. There is only one propagative branch for $m$ . In the limit of $\alpha\ll1$, we have :
\begin{equation}
	m=\pm\left(m_{0}+i\frac{\alpha n^{2}}{2 k\left(c-\overline{u}\right)^{3}\mathrm{Fr}^{2}m_{0}}\right)+o\left(\alpha\right)
\end{equation}
where 
\begin{equation}
 	m_{0}=\sqrt{\frac{n^{2}}{\left(c-\overline{u}\right)^{2}\mathrm{Fr}^{2}}-k^{2}}.
\end{equation}
is the positive inviscid solution for a purely propagating unattenuated wave. We assume that the wave remains propagating such that $\left(c-\overline{u}\right)^2 \mathrm{Fr}^2 k^2 <n^{2}$ is always verified. The sign depend on the direction of propagation of the considered wave. We can extract an attenuation length scale for the wave : 
\begin{equation}
	L_{\alpha}=\frac{2\left|k\right|\left|c-\overline{u}\right|^{3}\mathrm{Fr}^2 m_{0}}{\alpha n^{2}}.
\end{equation}

The definition of the vertical group velocity of an inviscid wave is naturally defined by $w_{g}=-k/\partial_{c}m_{0}$. We extend this definition to attenuated waves : 
\begin{equation}
	w_{g}\equiv\frac{k}{\partial_{c}\mathbb{Re}\left[m\right]}.\label{eq:GrpVel}
\end{equation}
In the limit $\alpha\ll1$, it leads to the inviscid group velocity : 
\begin{equation}
\left|w_{g}\right|=\frac{\left|k\right|m_{0}}{k^{2}+m_{0}^{2}}\left|c-\overline{u}\right|.
\end{equation}
The expression of the group velocity will be useful later to compare the time scale of evolution of the mean flow with the time scale of the propagation of the wave.

We can now consider back the equation for amplitude of the wave (\ref{eq:AmpEQGEN}) in the limit $\alpha\ll1$. It simplifies into :

\begin{equation}
	\partial_{Z}\left(\frac{\phi_{0}^{2}\left(c-\overline{u}\right)^{2}}{m_{0}}\right)=0.
\end{equation}

\subsection{Dispersion relation for $\alpha=0$ and $\mathrm{Re}<\infty$ :}

Without Newtonian cooling, the linear wave obeys the following dispersion relation locally : 
\begin{equation}
	\mathrm{Fr}^{2}\left(c-\overline{u}\right)^{2}\left(1-i\frac{\left(k^{2}+m^{2}\right)}{\mathrm{Re}k\left(c-\overline{u}\right)}\right)\left(k^{2}+m^{2}\right)=n^{2}.
\end{equation}
This an 4th\-- order equation. There is two branches for $m$. A propagative one and a boundary layer one. In the limit of $\mathrm{Re}\gg1$, we have :
\begin{align}
	m_{\mathrm{w}} 	& =-\text{sign}\left(\left(c-\overline{u}\right)k \right)\left(m_{0}+\frac{i n^{4}}{2k\left(c-\overline{u}\right)^{5}\mathrm{Fr}^{4}\mathrm{Re}m_{0}}\right)+o\left(\mathrm{Re}^{-1}\right),\\
	m_{\mathrm{bl}} 	& =\left(\text{sign}\left(\left(c-\overline{u}\right)k \right)-i\right)\sqrt{\frac{\left|c-\overline{u}\right|\left|k\right|\mathrm{Re}}{2}}+o\left(\sqrt{\mathrm{Re}}\right).
\end{align}
We assume that the waves are always non\--evanescent (i.e. $\left(c-\overline{u}\right)^2\mathrm{Fr}^2k^2<n^{2}$). $m_{\mathrm{w}}$ corresponds to the propagating branch. Its attenuation length scale is given by 
\begin{equation}
	L_{\mathrm{Re}}=\frac{2 \left|k\right| \left|c-\overline{u}\right|^5 \mathrm{Fr}^4\mathrm{Re}m_{0}}{n^{4}}.
\end{equation}
$m_{\mathrm{bl}}$ correspond to the boundary layer branch. The existence of this supplementary branch allows the matching of the propagative waves with the bottom boundary condition. The characteristic length of this boundary layer is given by :
\begin{equation}
	\delta_{\mathrm{Re}}=\sqrt{\frac{2}{\left|c-\overline{u}\right|\left|k\right|\mathrm{Re}}}.
\end{equation}

Using the definition of the group velocity in (\ref{eq:GrpVel}), we have at leading order in $\mathrm{Re}^{-1}$ :
\begin{align}
	\left|w_{g,\mathrm{w}}\right| 	& = \frac{\left|k\right|m_{0}}{k^{2}+m_{0}^{2}}\left|c-\overline{u}\right|\\
	\left|w_{g,\mathrm{bl}}\right|   	& = \sqrt{\frac{8\left|c-\overline{u}\right|\left|k\right|}{\mathrm{Re}}}
\end{align}

We can now consider back the equation for amplitude of the wave (\ref{eq:AmpEQGEN}) in the limit $\mathrm{Re}\gg1$. For $m_{\mathrm{w}}$, we have at leading order in $\mathrm{Re}^{-1}$:
\begin{equation}
	\partial_{Z}\left(\frac{\phi_{0,\mathrm{w}}^{2}\left(c-\overline{u}\right)^{2}}{m_{0}}\right)=0
\end{equation}
and for $m_{\mathrm{bl}}$, we have at leading order in $\mathrm{Re}^{-1}$:
\begin{equation}
	\partial_{Z}\left(\phi_{0,\mathrm{bl}}^{2}\left|c-\overline{u}\right|^{7/2}\right)=0
\end{equation}

\section{Generation by a moving boundary and Reynold stress}

\subsection{Boundary motion and boundary condition}

In this subsection, we apply the boundary condition at the bottom interface to compute the amplitude at the level of the bottom asymptotically flat boundary. We inject the WKB expansion (\ref{eq:WKB_Exp}) into the boundary conditions (\ref{eq:BC_1}-\ref{eq:BC_3})  and perform the wave and mean\--flow decomposition in section in section \ref{sec:Zonally} in order to get the bottom boundary conditions for the mean\--flow $\overline{u}$ and the slowly varying amplitudes of the leading order wave field. Wee need to consider the complete solution for the wave field, including both the propagating part and the boundary layer part ignoring the non\--physical divergent contributions. We consider a single propagative mode for the bottom boundary: $h_{b}\left(x,t\right)=\mathbb{R}\mathrm{e}\left[e^{i\left(ct-kx\right)}\right]$. We also need to assume how the bottom boundary velocity. We will consider two case here: either the bottom boundary is a fixed sine shape translating horizontally and $\mathbf{u}_{h_{b}}=c\mathbf{e}_{x}$ , either the bottom boundary is oscillating and then $\mathbf{u}_{h_{b}}=\partial_{t}h_{b}\,\mathbf{e}_{z}$ .

\subsubsection{Case of a translating boundary}
In this case, we consider only the free\--slip boundary condition. With a no\--slip boundary condition, the main effect of the bottom boundary would be the emergence of a strong viscous boundary layer for the shear mean\--flow. In the limit of asymptotically flat bottom boundary, it will not take long for viscous layer to be larger than the bottom boundary amplitude. After that internal gravity waves can no longer be emitted and all the theory introduced in this paper is obsolete. We are interested in the free-slip boundary condition in order to analyze high resolution internal gravity waves numerical simulations using large effective viscosities.
\begin{align}
	\partial_{Z}\overline{u} & 	= 	0 																				 		\\
	\phi_{0,\mathrm{w}}      &  = 	im_{\mathrm{w}}\frac{k^2-m_{\mathrm{bl}}^{2}}{m_{\mathrm{w}}^{2}-m_{\mathrm{bl}}^{2}} 	\\
	\phi_{0,\mathrm{bl}}     &  =	im_{\mathrm{bl}}\frac{k^2-m_{\mathrm{w}}^{2}}{m_{\mathrm{bl}}^{2}-m_{\mathrm{w}}^{2}}   
\end{align}

\textbf{High Reynolds limit}
In the limit of $\mathrm{Re}\gg1$, we have 
\begin{align}																			 		\\
	\phi_{0,\mathrm{w}}      &  = 	-\text{sign}\left(\left(c-\overline{u}\right)k\right)\left(im_{0}+\frac{k^{4}+3m_{0}^{4}}{2\left(c-\overline{u}\right)km_{0}\mathrm{Re}}\right) +o\left(\mathrm{Re}\right)	\\
	\phi_{0,\mathrm{bl}}     &  =	\left(1-\text{sign}\left(\left(c-\overline{u}\right)k\right)i\right)\frac{m_{0}^{2}-k^{2}}{\sqrt{\left|c-\overline{u}\right|\left|k\right|\mathrm{Re}}}
\end{align}

\subsection{Case of a oscillating boundary}
In this case, we consider only the no\--slip boundary condition. This case is considered only to compare with experimental set\--up where the boundary condition is no\--slip.
\begin{align}
	\overline{u} 			&  = 	0 																						\\
	\phi_{0,\mathrm{w}} 	&  = 	ic\frac{m_{\mathrm{w}}m_{\mathrm{bl}}}{m_{\mathrm{bl}}-m_{\mathrm{w}}} 					\\
	\phi_{0,\mathrm{bl}} 	&  = 	ic\frac{m_{\mathrm{w}}m_{\mathrm{bl}}}{m_{\mathrm{w}}-m_{\mathrm{bl}}} 		
\end{align}

\textbf{High Reynolds limit}
In the limit of $\mathrm{Re}\gg1$, we have 
\begin{align}																			 		\\
	\phi_{0,\mathrm{w}}      &  = 	-\text{sign}\left(\left(c-\overline{u}\right)k\right)\left(icm_{0}-\frac{k^{4}+2m_{0}^{2}k^{2}+3 m_{0}^{2}}{2km_{0}\mathrm{Re}} \right)-\left(\text{sign}\left(\left(c-\overline{u}\right)k\right)-i\right)m_{0}^{2}\sqrt{\frac{\left|c\right|}{2\left|k\right|\mathrm{Re}}}	\\
	\phi_{0,\mathrm{bl}}     &  =	\text{sign}\left(\left(c-\overline{u}\right)k\right)\left(icm_{0}-\frac{k^{4}+2m_{0}^{2}k^{2}+3 m_{0}^{2}}{2km_{0}\mathrm{Re}} \right)+\left(\text{sign}\left(\left(c-\overline{u}\right)k\right)-i\right)m_{0}^{2}\sqrt{\frac{\left|c\right|}{2\left|k\right|\mathrm{Re}}}	
\end{align}

\subsection{Computation of the Reynold stress}
\begin{align}
	\overline{u'w'}&=&	-\frac{\left(c-\overline{u}\right)^{2}k}{2}
						\mathbb{R}\mathrm{e}\left[\left(\frac{\phi_{0,\mathrm{w}}}{m_{\mathrm{w}}}e^{i\phi_{\mathrm{w}}}+\frac{\phi_{0,\mathrm{bl}}}{m_{\mathrm{bl}}}e^{i\varphi_{\mathrm{bl}}}\right)
						\left(\phi_{0,\mathrm{w}}^{*}e^{-i\varphi_{\mathrm{w}}^{*}}+\phi_{0,\mathrm{bl}}^{*}e^{-i\varphi_{\mathrm{bl}}^{*}}\right)\right]\\
	\overline{u'w'}&=& 	-\frac{\left(c-\overline{u}\right)^{2}k}{2}\left(
						\left|\frac{\phi_{0,\mathrm{w}}}{m_{\mathrm{w}}}\right|^{2}\mathbb{R}\mathrm{e}\left[m_{\mathrm{w}}\right]e^{-2\mathbb{I}\mathrm{m}\left[\varphi_{\mathrm{w}}\right]}
						+\left|\frac{\phi_{0,\mathrm{bl}}}{m_{\mathrm{bl}}}\right|^{2}\mathbb{R}\mathrm{e}\left[m_{\mathrm{bl}}\right]e^{-2\mathbb{I}\mathrm{m}\left[\varphi_{\mathrm{bl}}\right]}+\cdots\right.\\
				   & &	\left.+\mathbb{R}\mathrm{e}\left[\frac{\phi_{0,\mathrm{w}}^{*}\phi_{0,\mathrm{bl}}}{m_{\mathrm{w}}^{*}m_{\mathrm{bl}}}\left(m_{\mathrm{w}}^{*}+m_{\mathrm{bl}}\right)\right]
						\mathbb{R}\mathrm{e}\left[e^{i\left(\varphi_{\mathrm{bl}}-\varphi_{\mathrm{w}}^{*}\right)} \right] 
	                	-\mathbb{I}\mathrm{m}\left[\frac{\phi_{0,\mathrm{w}}^{*}\phi_{0,\mathrm{bl}}}{m_{\mathrm{w}}^{*}m_{\mathrm{bl}}}\left(m_{\mathrm{w}}^{*}+m_{\mathrm{bl}}\right)\right]
						\mathbb{I}\mathrm{m}\left[e^{i\left(\varphi_{\mathrm{bl}}-\varphi_{\mathrm{w}}^{*}\right)} \right] \right)
\end{align}
\subsubsection{Case of a translating boundary}

\begin{align}
	\overline{u'w'}&=& -\frac{\left(c-\overline{u}\right)^{2}k}{2\left|m_{\mathrm{bl}}^{2}-m_{\mathrm{w}}^{2}\right|^{2}}\left(
						\frac{m_{0}\left(Z\right)}{m_{0}}\left|k^2-m_{\mathrm{bl}}^{2}\right|^{2}\mathbb{R}\mathrm{e}\left[m_{\mathrm{w}}\right]e^{-2\mathbb{I}\mathrm{m}\left[\varphi_{\mathrm{w}}\right]}
						+\left(\frac{\left|c-\overline{u}\left(Z\right)\right|}{\left|c-\overline{u}\right|}\right)^{3/2}\left|k^{2}-m_{\mathrm{w}}^{2} \right|^{2}\mathbb{R}\mathrm{e}\left[m_{\mathrm{bl}}\right]e^{-2\mathbb{I}\mathrm{m}\left[\varphi_{\mathrm{bl}}\right]}+\cdots\right.\\
				   & &	-\mathbb{R}\mathrm{e}\left[\left(k^{2}-m_{\mathrm{bl}}^{*2}\right)\left(k^{2}-m_{\mathrm{w}}^{2}\right)\left(m_{\mathrm{w}}^{*}+m_{\mathrm{bl}}\right)\right]
						\mathbb{R}\mathrm{e}\left[e^{i\left(\varphi_{\mathrm{bl}}-\varphi_{\mathrm{w}}^{*}\right)} \right]\\
				   & & 	\left.+\mathbb{I}\mathrm{m}\left[\left(k^{2}-m_{\mathrm{bl}}^{*2}\right)\left(k^{2}-m_{\mathrm{w}}^{2}\right)
				   		\left(m_{\mathrm{w}}^{*}+m_{\mathrm{bl}}\right)\right]
						\mathbb{I}\mathrm{m}\left[e^{i\left(\varphi_{\mathrm{bl}}-\varphi_{\mathrm{w}}^{*}\right)} \right] \right)
\end{align}

In the limit of large Reynold number, $\mathrm{Re}\gg1$, we have 







\bibliographystyle{plain}
\bibliography{Bibliographie}

\end{document}
